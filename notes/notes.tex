\documentclass[a4paper]{article}

%\usepackage[numbers]{natbib}
\usepackage[top=2.5cm,bottom=2.5cm]{geometry}
\usepackage{amsmath}
\usepackage{graphicx}
\usepackage{hyperref}


%\bibliographystyle{naturemagurl}
%\bibliographystyle{plainnat}

\newcommand{\evec}[2]{\phi^{(#2)}_{#1}}
\newcommand{\subK}{K^{[i^{*}]}}

\title{Notes on Minimum Spanning Tree spectral decomposition and preconditioning}
\author{Jacob Stevenson}
\date{\today}

\begin{document}
\maketitle

\section{eigenvalues and vectors}

The eigenvalues $\lambda_n$ and eigenvectors $\phi^{(n)}_i$ orthonormality

\begin{equation}
\sum_i \phi^{(n)}_i \pi_i \phi^{(m)}_i = \delta_{nm}
\end{equation}

They also satisfy the eigenvector equation
\begin{equation}
\sum_i K_{ij} \evec{i}{n}  = \lambda_n \evec{j}{n}
\end{equation}

This can be written more compactly as

\begin{align}
\Phi^{T} \Pi \Phi = I \\
K \Phi = \Phi \Lambda
\end{align}
where $\Pi$ and $\Lambda$ are diagonal matrices with $\pi_i$ and $\lambda_i$ along the diagonals.
The inverse of $\Phi$ is $\Phi^{T} \Pi$.
We can write K from the spectral decomposition
\begin{align}
K_{ij} &= \sum_n \lambda_n \evec{i}{n} \evec{j}{n} \pi_j \\
K &= \Phi \Lambda \Phi^T \Pi
\end{align}
The pseudo-inverse of $K$ can be written as
\begin{equation}
K^{+} = \Phi \Lambda^{-1} \Phi^T \Pi
\end{equation}
This is not the real inverse because the first eigenvector is zero.  But it does satisfy 
\begin{align}
K K^{+} K &= K \\
K^{+} K K^{+} &= K^{+}
\end{align}
and maybe (but I haven't checked it)
\begin{align}
(K K^{+})^T &= K K^{+} \\
(K^{+} K)^T &= K^{+} K
\end{align}

\section{submatrix spectrum}
We ultimately want to construct a preconditioner for a submatrix of $K$ denoted $\subK{}$ which is $K$
with row and column $i^{*}$ deleted.  This matrix is used to calculate mean first passage times from all the nodes
into $i^{*}$ via
\begin{equation}
\subK{} \tau_{mfpt} = (1, \dots, 1)^T
\end{equation}
The preconditioner will be an approximation to the inverse of $\subK{}$ which we will hopefully construct from 
the spectrum of $K$.

We will proceed by altering the original problem so that the energy of the node to be removed $E_{i^{*}}$
is much less than all other energies in the problem.  The spectrum of $K$ is then computed.

We will try to show that the eigenvectors of $\subK{}$ are the same as the eigenvectors of $K$ with 
the $i^{*}$th element removed. The eigenvector corresponding to the zero eigenvalue is discarded.

We are taking the limit $E_{i^{*}} \to -\infty$, so we will define a small parameter $p = \exp{\beta E_{i^{*}}}$.
Numerically it appears that
\begin{align}
\lambda_{i} &\sim 1 \\
\evec{i^{*}}{n} &\sim p \\
\evec{i\ne i^{*}}{n} &\sim 1 \text{  (except, possibly, for n=1)} \\
\pi_{i^{*}} &\sim 1 \\
\pi_{i\ne i^{*}} &\sim p \\
K_{i^{*} j} &\sim p \\
K_{i\ne i^{*}, j} &\sim 1
\end{align}

First show that they are orthonormal.
\begin{align}
\sum_{i\ne i^{*}} \evec{i}{n} \pi_i \evec{i}{m} &= \delta_{nm} - \evec{i^{*}}{n} \evec{i^{*}}{m} \pi_{i^{*}} \\
& \sim \delta_{nm} - p^2
\end{align}
Thus they are orthonormal for small enough p.  We now show that they satisfy the eigenvector equation
\begin{align}
\sum_{j\ne i^{*}} K_{ij} \evec{j}{n}  &=
\lambda_n \evec{i}{n} - K_{i i^{*}} \evec{i^{*}}{n} \\
& \sim \lambda_n \evec{i}{n} - p
\end{align}
Similarly, it can be shown that
\begin{align}
\sum_{i\ne i^{*}} 
\sum_{j\ne i^{*}} 
\evec{i}{m} \pi_{i}
K_{ij} \evec{j}{n} 
&= \lambda_n \delta_{nm} + \evec{i^{*}}{n} \evec{i^{*}}{m} \pi_{i^{*}} (K_{i^{*} i^{*}} - \lambda_n - \lambda_m)
\\
& \sim \lambda_n \evec{i}{n} + p^2 (p - 1)
\end{align}



\end{document}
